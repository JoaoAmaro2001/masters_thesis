%%%%%%%%%%%%%%%%%%%%%%%%%%%%%%%%%%%%%%%%%%%%%%%%%%%%%%%%%%%%%%%%%%%%%%%%
%                                                                      %
%     File: Thesis_Glossary.tex                                        %
%     Tex Master: Thesis.tex                                           %
%                                                                      %
%     Author: Andre C. Marta                                           %
%     Last modified : 27 Feb 2024                                      %
%                                                                      %
%%%%%%%%%%%%%%%%%%%%%%%%%%%%%%%%%%%%%%%%%%%%%%%%%%%%%%%%%%%%%%%%%%%%%%%%
%
% The definitions can be placed anywhere in the document body
% and their order is sorted by <key> automatically when
% calling makeindex in the makefile
%
% ----------------------------------------------------------------------
% To create a glossary entry, use the following syntax:
%
% \newglossaryentry{<label>}{name={<key>}, description={<value>}}
%
% where the parameters are:
% <label> is the label of the entry,
% <key> is the acronym to be defined by the glossary entry (in lowercase, preferably)
% <value> is the actual definition of the current term
%
% To produce the desired term in the document, that will be replaced by
% the user-defined in the output, use the following syntax:
%
% \gls{<label>}
%
% ----------------------------------------------------------------------
% To create a acronym entry, use the following syntax:
%
% \newacronym{⟨label⟩}{⟨abbrv⟩}{⟨full⟩}
%
% where the parameters are:
% <label> is the label of the entry,
% <abbrv> is the acronym,
% <full> is the definition of the acronym
%
% To produce the desired term in the document, that will be replaced by
% the user-defined in the output, use one of the following syntaxes:
%
% \acrlong{<label>}
% \acrshort{<label>}
% \acrfull{<label>}
%
% ----------------------------------------------------------------------
% By default, only those entries defined in the main document using the
% commands above will be displayed in the glossary (list of acronyms),
% unless the command \glsaddall is used,
% ----------------------------------------------------------------------

% The order of the definitions below is irrelevant
% since the glossary is automatically ordered alphabetically

\newacronym{laura}{LAURA}{Local Autoregressive Average}
\newacronym{loreta}{LORETA}{Low Resolution Electromagnetic Tomography}
\newacronym{sloreta}{sLORETA}{Standardized Low Resolution Electromagnetic Tomography}
\newacronym{eloreta}{eLORETA}{Exact Low Resolution Electromagnetic Tomography}
\newacronym{mne}{MNE}{Minimum Norm Estimate}
\newacronym{wmn}{WMN}{Weighted Minimum Norm}
\newacronym{erds}{ERD/ERS}{Event-related desynchronization/synchronization}
\newacronym{esi}{ESI}{Electrical Source Estimation}
\newacronym{erp}{ERP}{Event-related Potential}
\newacronym{lpp}{LPP}{Late Positive Potential}
\newacronym{epn}{EPN}{Early Posterior Negativity}
\newacronym{mmn}{MMN}{Mismatch negativity}
\newacronym{ern}{ERN}{Error-related Negativity}
\newacronym{eeg}{EEG}{Electroencephalography}
\newacronym{mri}{MRI}{Magnetic Resonance Imaging}
\newacronym{fmri}{fMRI}{Fucntional Magnetic Resonance Imaging}
\newacronym{fnirs}{fNIRS}{Functional Near-Infrared Spectrometry}
\newacronym{sam}{SAM}{Self-Assessment Manikin}
\newacronym{sdi}{SDI}{Spatial Data Infrastructure}
\newacronym{bids}{BIDS}{Brain Imaging Data Standard}
\newacronym{hdeeg}{HD-EEG}{High-density Encephalography}
\newacronym{art}{ART}{Attention Restoration Theory}
\newacronym{srt}{SRT}{Stress Reduction Theory}
\newacronym{sd}{SD}{Standard Deviation}
\newacronym{pare}{PARE}{Polar Average Reference Effect}
\newacronym{fir}{FIR}{Finite Impulse Response}
\newacronym{ad}{AD}{Alzheimer’s disease}
\newacronym{mci}{MCI}{Mild Cognitive Impairment}
\newacronym{adhd}{ADHD}{Attention Deficit Hyperactivity Disorder}
\newacronym{vr}{VR}{Virtual Reality}
\newacronym{ai}{AI}{Artificial Intelligence}
\newacronym{bci}{BCI}{Brain-computer interface}
\newacronym{gui}{GUI}{Graphical User Interface}
\newacronym{bem}{BEM}{Boundary Element Method}
\newacronym{fem}{FEM}{Finite Element Method}
\newacronym{fdm}{FDM}{Finite Difference Method}
\newacronym{ba}{BA}{Brodmann Area}
\newacronym{roi}{ROI}{Region of Interest}
\newacronym{ecd}{ECD}{Equivalent Current Dipole}
\newacronym{music}{MUSIC}{Multiple Signal Classification}
\newacronym{rapmusic}{RAP-MUSIC}{Recursively Applied Multiple Signal Classification}
\newacronym{dspm}{dSPM}{Dynamic Statistical Parametric Mapping}
\newacronym{sslofo}{SSLOFO}{Standardized shrinking LORETA-FOCUSS}
\newacronym{regea}{REGEA}{Regional Activity Estimation} 
\newacronym{focuss}{FOCUSS}{Focal Underdetermined System Solution}
\newacronym{snr}{SNR}{Signal-to-Noise Ratio}
\newacronym{tfce}{TFCE}{Threshold-Free Cluster Enhancement}
\newacronym{dicom}{DICOM}{Digital Imaging and Communications in Medicine}
% -----------------------------------------------(manual above)
%\newacronym{⟨label⟩}{⟨abbrv⟩}{⟨full⟩}
%\newacronym{⟨label⟩}{⟨abbrv⟩}{⟨full⟩}
%\newacronym{⟨label⟩}{⟨abbrv⟩}{⟨full⟩}

% ----------------------------------------------------------------------
% displays all entries (even those unused with commands \acrlong/short/full)
\glsaddall

% ----------------------------------------------------------------------
% vertical aligment of acronyms' long names
\setlength\LTleft{0pt}
\setlength\LTright{0pt}
\setlength\glsdescwidth{1.0\hsize}
