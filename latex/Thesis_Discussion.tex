%%%%%%%%%%%%%%%%%%%%%%%%%%%%%%%%%%%%%%%%%%%%%%%%%%%%%%%%%%%%%%%%%%%%%%%%
%                                                                      %
%     File: Thesis_Discussion.tex                                     %
%     Tex Master: Thesis.tex                                           %
%                                                                      %
%     Author: João Amaro                                               %
%     Last modified :  7 nov 2024                                      %
%                                                                      %
%%%%%%%%%%%%%%%%%%%%%%%%%%%%%%%%%%%%%%%%%%%%%%%%%%%%%%%%%%%%%%%%%%%%%%%%

\chapter{Discussion}
\label{chapter:discussions}

\section{Behavioral data}

As expected, valence and arousal ratings followed what is described in the literature - natural stimuli are reported to elicit more positive  emotions and less arousal when compared to urban-built stimuli \cite{mahamaneNaturalCategorizationElectrophysiological2020, mavrosMobileEEGStudy2022, mavrosAttenuatingSubjectiveCrowding2023}. It is important to address that the difference across conditions for the valence ratings showed  a much bigger effect size when compared to the arousal ratings \cite{sawilowskyNewEffectSize2009}. Such a difference might be due to the task itself as it is harder to elicit stronger magnitudes of emotions, both positive and negative, by showing videos on a screen. 

One option to facilitate the study of arousal states on EEG in the context of neurourbanism is to move studies from the indoor lab to the outdoor environment. In fact, due to its portability EEG is a technique which can be implemented both in the indoor and outdoor scenarios, allowing for the possibility of comparative analysis.

\section{Task considerations}

The findings of this study are highly dependent on the accurate characterization of the stimuli, which is critical for creating meaningful conditions for comparison. However, two significant limitations in this area warrant discussion. First, the stimuli in this study were categorized into conditions based on subjective ratings derived from the entire video, despite only the first frame being used as the representative stimulus. This discrepancy could introduce biases and reduce the validity of the condition-specific comparisons.

Additionally, the potential confounding factor of crowdedness complicates the interpretation of results. While trials categorized as "high crowdedness" (categories "B" and "D") were initially considered for exclusion to avoid their potential influence, they were ultimately retained to maintain an adequate trial count per condition, which otherwise would have been limited to only 16 trials. This point should be carefully considered, as research has demonstrated that the crowdedness factor independently affects self-perceived valence and arousal, thereby potentially modulating the observed results \cite{mavrosAttenuatingSubjectiveCrowding2023, mavrosMobileEEGStudy2022}.

Moreover, the use of a simple "urban vs. nature dichotomy" \cite{karmanovAssessingRestorativePotential2008} oversimplifies the complexities of environmental characterization. This binary approach may obscure more nuanced environmental features that contribute to the observed neural and emotional responses. A more sophisticated characterization of the videos could be achieved through advanced methods such as image segmentation tools \cite{shenBackVillageAssessing2024a}, which would allow for a deeper understanding of the specific environmental attributes influencing the data. These improvements would enhance the robustness of future studies in this field.

\section{ERP results}

This work is the first to describe a HD-EEG methodology to investigate the neural correlates of urban scenery emotional processing. A recent systematic review identified EEG as the most commonly used neuroscience technique in this domain. However, prior studies predominantly utilized low-channel EEG systems, with the highest number of channels being 64 in a single study \cite{ancoraCitiesNeuroscienceResearch2022}. This prevalence of low-channel EEG is likely due to the increased focus on frequency-domain analysis, particularly within the alpha band, and to a lesser extent, the theta and beta bands.

One objective of employing an ERP methodology in this work was to determine whether images of urban scenarios could elicit unconscious emotional processing. If such processing occurs, it would suggest that specific visual elements in urban scenery trigger distinct neural activities. In fact, early ERP components can be particularly sensitive to low-level visual features the images, such as color and luminosity, and as such there needs to be further control of these possible confounds \cite{grassiniProcessingNaturalScenery2019}.

The mass univariate approach used here offers an unbiased means of analyzing EEG data without prior assumptions. However, coupling this approach with multiple comparison correction methods, such as the TFCE correction, introduces a level of conservatism that makes detecting small effect sizes challenging \cite{luckHowGetStatistically2017}. While no statistically significant differences between conditions were found, topographical maps revealed the greatest variability in frontal electrodes, suggesting a potential region of interest for future investigations.

Despite the lack of significant differences, a pronounced posterior negativity was observed for urban stimuli, and the variability in EPN peak latency was much lower compared to the LPP. This indicates that the EPN may be a more robust component in this context. Interestingly, behavioral data suggested that valence differences, which showed larger effect sizes, would align more strongly with the LPP component. This discrepancy may be explained by two factors. First, valence ratings were based on entire video sequences, likely reflecting long-lasting processes that extend beyond the one-second epoch used in this work. Thus, neural correlates underlying self-perceived valence may operate on a different timescale than ERPs, which capture millisecond-level processes. Second, the absence of a pronounced LPP component could be attributed to the nature of the task, which differs fundamentally from the oddball paradigm where the LPP was originally characterized \cite{hajcakEventRelatedPotentialsEmotion2010}.

The findings of Mahamane et al. (2020) provide additional context. Their study explored EEG differences between natural and urban-built stimuli, specifically looking for differential LPP patterns. However, their use of an oddball-task paradigm with defined standard and target stimuli—contrasts \cite{mahamaneNaturalCategorizationElectrophysiological2020}. This approach is fundamentally different from the one employed in this work where the same number of natural and urban-built videos were shown and the order was randomized. Moreover, their EEG setup featured only 14 channels, offering much lower spatial resolution. Despite these differences, they identified sustained late positivity in frontal channels, aligning with the higher frontal topographical varibalitity observed in this work. These results suggest that the topographical pattern of the LPP component may warrant reconsideration in future research.

\section{Source analysis results}

One limitation of this study was the constraints imposed by EGI's software on the way source data could be analyzed. The Reciprocity GUI, where the results are visualized, provides some information about the anatomical positions of the sources; however, this information is restricted to only a few areas. While it is possible to export the time-series of the dipoles, there is no intuitive way to map the head model dipoles to specific anatomical areas. This limitation makes it challenging to assess regions such as the amygdala, which has been shown to be more active in urban scenarios and might serve as one of the neural generators of the EPN \cite{grassiniProcessingNaturalScenery2019, kimHumanBrainActivation2010}.

Another important consideration is that the head models used in the study were not perfect, as they were subject to artifacts caused by the EEG-fMRI caps and headphones worn by participants. The latter created artifacts by slightly compressing the temporal regions of the scalp, potentially biasing dipoles projecting to these areas. Indeed, many dipoles exhibited unusually strong temporal projections. To address this issue, care was taken to exclude such dipoles from the analysis to minimize bias.

As this study represents one of the first applications of electrical source imaging (ESI) to EEG data in the context of urbanism, its findings require comparison with results from fMRI studies using similar methodologies. Notably, one fMRI study using a comparable paradigm found higher activation in the hippocampus and parahippocampal gyrus for urban stimuli compared to rural stimuli, aligning with the results of this study \cite{kimHumanBrainActivation2010}. Another similar fMRI study by the same group replicated these findings while also reporting increased activity in the anterior temporal pole for urban scenery compared to natural scenery \cite{kimFunctionalNeuroanatomyAssociated2010}. These observations corroborate the evidence that negative emotions are associated with urban environments and underscore the value of source estimation when using high-density EEG.

Interestingly, previous studies applying LORETA to the N170 component have identified BA38 as a region of maximal activation in contrasts between fear- and anger-inducing conditions \cite{tsolakiAgeinducedDifferencesBrain2017}. This study’s findings are consistent with these results, which is notable given the similarities between the N170 and EPN components in terms of polarity, time window, and scalp distribution \cite{luckOxfordHandbookEventrelated2013}. Most sources identified in this study were concentrated in ventral brain structures, particularly the temporal-occipital regions, areas typically involved in visual and emotional processing. However, these results are likely influenced by the sLORETA algorithm used for source estimation. While commonly used algorithms tend to identify similar source locations \cite{shiraziMoreReliableEEG2019}, testing alternative algorithms, such as LAURA, could help address the inherent variability in source estimation techniques. Furthermore, analyzing sources across the entire EPN time window, rather than focusing on a single time point, could provide a more comprehensive understanding of source dynamics and reduce reliance on algorithmic choices \cite{mahjooryConsistencyEEGSource2017}.

These findings also align with theories suggesting that the potency of negative affect reflects the rapid processing of aversive information by the amygdala and related structures \cite{rozenkrantsAffectiveERPProcessing2008}. This is further supported by prior studies using current source density and minimum norm analysis, which have indicated that emotional versus neutral differences in potential arise from true neuronal sources over occipito-temporo-parietal sites \cite{schuppStimulusNoveltyEmotion2006, junghoferFleetingImagesNew2001}.

Finally, one of the most compelling findings of this work is the lateralized contrast observed between the temporal sources in Broadmann areas 38 and 20. Specifically, the right temporal lobe exhibited stronger dipoles for urban stimuli, while the opposite was observed for natural stimuli. This novel effect has not been demonstrated in previous studies, suggesting a unique neural response pattern to urban and natural environments that warrants further exploration.

%Hippocampus and parahippocampus suggests there might be a memory component to these stimuli. These findings suggest that urban settings may engage the brain more intensively, particularly in regions associated with sensory integration and memory, while natural environments evoke more diverse and individualized patterns of neural activity.

% ----------------------------------------------------------------------
\section{Future Work}
\label{section:future_work}

To build upon the results of this study, several directions for future research can be explored. One key area is assessing the consistency of source estimation algorithms by comparing the performance of different algorithms. This could help establish the reliability of the source localization findings. Additionally, the stimuli presented in this task could be implemented in other experimental paradigms, such as the oddball paradigm,  which could provide further insights into how the choice of task affects the observed ERP components.

Expanding the analysis to include other time-windows and ERP components, such as the P1-N1 complex and the P3, could offer a more comprehensive understanding of the temporal dynamics of neural responses to these stimuli. Furthermore, applying methodologies focused on the time-frequency domain, particularly the ERS/ERD approach, holds promise. This is especially relevant since frequency bands, such as the alpha rhythm, have been shown to exhibit distinct characteristics under natural versus urban conditions, with greater alpha power in natural environments suggesting a state of relaxation.

Connectivity analysis also presents an exciting avenue for future exploration. By focusing on the connectivity aspects of the data, particularly in source space, it may be possible to identify distinct neural networks associated with the observed differences in source distribution. This approach, when coupled with source reconstruction, could identify the functional networks underlying emotional processing in environmental settings.
