%%%%%%%%%%%%%%%%%%%%%%%%%%%%%%%%%%%%%%%%%%%%%%%%%%%%%%%%%%%%%%%%%%%%%%%%
%                                                                      %
%     File: Thesis_Abstract.tex                                        %
%     Tex Master: Thesis.tex                                           %
%                                                                      %
%     Author: Andre C. Marta                                           %
%     Last modified :  4 Mar 2024                                      %
%                                                                      %
%%%%%%%%%%%%%%%%%%%%%%%%%%%%%%%%%%%%%%%%%%%%%%%%%%%%%%%%%%%%%%%%%%%%%%%%

\section*{Abstract}

% Add entry in the table of contents as section
\addcontentsline{toc}{section}{Abstract}

Neurourbanism is an emerging interdisciplinary field that leverages neuroscientific methodologies to inform urban planning and design. Among these methods, electroencephalography (EEG) stands out for its ability to provide real-time insights into how the brain perceives urban environments. A key application of EEG in this context involves the study of emotional responses through event-related potential (ERP) components, such as the early posterior negativity (EPN), primarily linked to stimulus arousal, and the late positive potential (LPP), primarily associated with stimulus valence.

This study aims to investigate whether these ERP components can differentiate between urban scenarios, specifically natural and urban-built stimuli. Additionally, electrical source estimation (ESI) was applied to these ERP components to identify their neuronal generators and explore how these differ across conditions.

High-density EEG data with 257 channels was collected from 30 adult participants during a passive viewing task, where they were exposed to image frames of various urban scenarios for the first second following stimulus onset. Structural MRI data was also acquired for some participants to enhance source localization.

Results revealed a robust EPN component, while the LPP component was absent in the dataset. Although no significant differences were observed in the time-domain analysis of the EPN between conditions, the urban condition exhibited a more pronounced posterior negativity, indicative of heightened stimulus arousal. Source-domain analysis identified increased neural activity in the anterior temporal and occipital regions, including the fusiform gyrus and parahippocampal areas. Additionally, the source analysis revealed also hemispherical differences in the neural generators of the EPN.

Overall, the findings of this study provide valuable insights into the neural correlates of emotional processing in response to urbanistic scenery. Additionally, they highlight the utility of source imaging as a powerful complement to conventional EEG analysis, offering a more nuanced understanding of the underlying neural mechanisms.

\vfill

\textbf{\Large Keywords:} EEG, ERP, LPP, EPN, source imaging, neurourbanism, emotions

