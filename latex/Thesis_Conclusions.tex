%%%%%%%%%%%%%%%%%%%%%%%%%%%%%%%%%%%%%%%%%%%%%%%%%%%%%%%%%%%%%%%%%%%%%%%%
%                                                                      %
%     File: Thesis_Conclusions.tex                                     %
%     Tex Master: Thesis.tex                                           %
%                                                                      %
%     Author: Andre C. Marta                                           %
%     Last modified :  4 Mar 2024                                      %
%                                                                      %
%%%%%%%%%%%%%%%%%%%%%%%%%%%%%%%%%%%%%%%%%%%%%%%%%%%%%%%%%%%%%%%%%%%%%%%%

\chapter{Conclusion}
\label{chapter:conclusions}

The results of this study provide a comprehensive overview of the differences in emotional and neural responses to natural and urban environments. Behavioral analyses revealed that participants rated natural environments as significantly more positive in valence and less arousing than urban-built environments. 

The ERP analysis provided further insights into these behavioral differences. Although no significant clusters were observed in the grand-average ERP after correcting for multiple comparisons, distinct neural patterns emerged when specific ERP components were analyzed. The EPN component, associated with visual attention and emotional salience, was stronger for urban stimuli and, in contrast, the LPP component lacked a clear sustained positivity, displaying much variability across participants and conditions. Despite visual observations suggesting component differences, statistical tests for peak amplitude, mean amplitude, and latency did not yield significant results after Bonferroni correction.

Source estimation further highlighted the contrasting neural mechanisms underlying responses to natural and urban environments. Urban stimuli elicited more focalized brain activity, particularly in the left temporal region, which is associated with processing complex sensory and cognitive demands typical of urban settings. In contrast, natural stimuli activated broader and more distributed regions, including the right temporal and occipital areas. Analysis of Brodmann Areas indicated prominent activity in anterior temporal regions (BA38 and BA20) for both conditions, while also engaged in occipital and parahippocampal regions.


Overall, these findings reinforce the evidence that the neural correlates of emotional processing exhibit distinct activity patterns in response to natural and urban-built stimuli. Moreover, they highlight the need for further investigation into the neural generators of the EPN component, underscoring the value of source imaging in enhancing EEG's role as a veritable neuroimaging technique.

