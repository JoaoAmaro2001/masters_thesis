%%%%%%%%%%%%%%%%%%%%%%%%%%%%%%%%%%%%%%%%%%%%%%%%%%%%%%%%%%%%%%%%%%%%%%%%
%                                                                      %
%     File: Thesis_Resumo.tex                                          %
%     Tex Master: Thesis.tex                                           %
%                                                                      %
%     Author: Andre C. Marta                                           %
%     Last modified :  4 Mar 2024                                      %
%                                                                      %
%%%%%%%%%%%%%%%%%%%%%%%%%%%%%%%%%%%%%%%%%%%%%%%%%%%%%%%%%%%%%%%%%%%%%%%%

\section*{Resumo}

% Add entry in the table of contents as section
\addcontentsline{toc}{section}{Resumo}

O neurourbanismo é uma área interdisciplinar emergente que utiliza metodologias neurocientíficas para informar o planeamento e o design urbano. Entre estas metodologias, a eletroencefalografia (EEG) destaca-se pela sua capacidade de fornecer informação de alta resoluç\~{a}o temporal sobre como o cérebro processa os ambientes urbanos. Uma aplicação chave do EEG neste contexto envolve o estudo das respostas emocionais através de componentes dos "event-related potentials" (ERP), como por exemplo a "early posterior negativity" (EPN), associada principalmente à excitação induzida por estímulos, e o "late positive potential" (LPP), associado principalmente ao prazer induzido por estímulos.

Este estudo tem como objetivo investigar se estes componentes do ERP conseguem diferenciar entre cenários de cidade diferentes, especificamente entre estímulos naturais e urbanos. Adicionalmente, uma técnica de "electrical source imaging"(ESI) foi aplicada a a uma destas componentes para identificar os seus geradores neuronais e explorar como estes diferem entre as condições definidas.

Foram recolhidos dados de EEG de alta densidade com 257 canais de 30 participantes adultos durante uma tarefa de visualização passiva, onde foram expostos a imagens de vários cenários urbanos durante o primeiro segundo após a apresentação do estímulo. Para alguns participantes, também foram adquiridos dados de ressonância magnética estrutural (MRI) para melhorar a localização das fontes.

Os resultados revelaram uma componente EPN robusta, enquanto o componente LPP não foi visível nos dados. Embora não tenham sido observadas diferenças significativas para o EPN entre condições, a condição urbana exibiu uma negatividade posterior mais acentuada, indicativa de uma maior excitação provocada pelos estímulos urbanos. A análise referente ao ESI identificou uma atividade neural aumentada nas regiões temporais anteriores e occipitais, incluindo o giro fusiforme e as áreas parahipocampais. Adicionalmente, esta análise também revelou diferenças hemisféricas nos geradores neuronais do EPN.

No geral, os resultados deste estudo forneceram  evid\^{e}ncia sobre como a atividade neuronal se relaciona com o processamento emocional induzido por cenários urbanísticos. Adicionalmente, destacam também a utilidade da técnica de ESI como uma ferramenta complementar bastante útil à análise convencional de EEG, sendo capaz de dar informação mais detalhada sobre os mecanismos neuronais subjacentes.

\vfill

\textbf{\Large Palavras-chave:} EEG, ERP, espaços urbanos, emoção, fontes neuronais

