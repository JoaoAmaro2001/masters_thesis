%%%%%%%%%%%%%%%%%%%%%%%%%%%%%%%%%%%%%%%%%%%%%%%%%%%%%%%%%%%%%%%%%%%%%%%%
%                                                                      %
%     File: Thesis_Introduction.tex                                    %
%     Tex Master: Thesis.tex                                           %
%                                                                      %
%     Author: Andre C. Marta                                           %
%     Last modified :  4 Mar 2024                                      %
%                                                                      %
%%%%%%%%%%%%%%%%%%%%%%%%%%%%%%%%%%%%%%%%%%%%%%%%%%%%%%%%%%%%%%%%%%%%%%%%

\chapter{Introduction}
\label{chapter:introduction}

%%%%%%%%%%%%%%%%%%%%%%%%%%%%%%%%%%%%%%%%%%%%%%%%%%%%%%%%%%%%%%%%%%%%%%%%
\section{Neurourbanism}
\label{section:neurourbanism}

\subsection{Increasing Urbanization}

The current worldwide trend toward the adoption of an urban-predominant lifestyle requires a concomitant degree of research to better understand the effects that city life can have on the mental health of its individuals and communities \cite{adliNeurourbanismNewDiscipline2017}. In fact, reviews on the topic have already emphasized an association of urban life with mental disorders, which are more prevalent in cities and highly dependent on social, economic, and environmental factors \cite{galeaURBANHEALTHEvidence2005, ventriglioUrbanizationEmergingMental2021}. Additionally, several studies have shown that natural environments contribute to the increase of positive emotions, improve attention, decrease stress, and are overall preferred to urban scenery \cite{kondoDoesSpendingTime2018, meidenbauerGradualDevelopmentPreference2019, ohlyAttentionRestorationTheory2016}. These findings also support several psychological theories, such as the Attention Restoration Theory (ART) \cite{kaplanRestorativeBenefitsNature1995}, which claims exposure to natural environments overcomes mental fatigue, and the Stress Reduction Theory (SRT) \cite{ulrichAestheticAffectiveResponse1983}, which claims nature settings induce calmness and reduce stress. Both these theories conform with the biophilia hypothesis, which claims humans have an evolutionary affinity towards nature \cite{jimenezAssociationsNatureExposure2021}. 
Given this knowledge, increase attention has been drawn towards the field of neurourbanism which aims towards the use of basic and clinical neuroscience research tools to investigate the effects of urban living on neurological processes and applying this knowledge to support a better urban planning and design framework \cite{adliNeurourbanismNewDiscipline2017,mondscheinNewDirectionsCognitiveEnvironmental2018}.

%Researchers are interested to understand what aspects of the built environment itself can improve mental and physical wellbeing (Saarloos et al., 2011; Guzman et al., 2021; Buttazzoni et al., 2021). \hl{and mavros 2023}. It is hoped that by understanding which characteristics of the ubran environment elicit more positive emotions, mental health prevalence in the cities can decrease and wellbeing can increase.

\subsection{Emotions in urbanism}

When people interact with their environment, they experience a full spectrum of emotions ranging from negative feelings to positive ones. This emotional response is crucial because emotions drive our major life decisions and behavioral choices, which in turn influence our pursuit of positive experiences and avoidance of negative ones - key factors in overall well-being. Importantly, these cognitive and emotional responses to our environment translate into real physiological effects on our health \cite{fredricksonCultivatingPositiveEmotions2000}.

Efforts to understand how urban environments influence emotional states have often relied on the valence-arousal framework. This framework conceptualizes emotions as states of vigilant readiness, plotted within a two-dimensional space defined by two coordinates: valence (a global measure of positivity or negativity associated with an emotional state) and arousal (the intensity or dynamism of the emotional state) \cite{langEmotionProbeStudies1995, nidalEEGERPAnalysis2014}. Russell’s circumplex model expanded on this framework by providing a systematic approach to mapping valence and arousal ratings onto specific emotional experiences, offering a comprehensive tool for emotional assessment \cite{russellCircumplexModelAffect1980}.

A key instrument for applying this framework is the Self-Assessment Manikin (\acrshort{sam}), a validated scale designed to standardize the measurement of valence, arousal, and dominance \cite{bradleyMeasuringEmotionSelfassessment1994}. More recent adaptations, such as the Affective Slider, have modernized the SAM to suit digital applications while maintaining its robustness and reliability \cite{betellaAffectiveSliderDigital2016}.

The strength of employing the valence-arousal framework lies in its ability to bridge psychological and physiological states. For example, arousal can be quantified through autonomic nervous system responses, such as changes in heart rate, skin temperature, or galvanic skin response, providing objective measures to complement subjective emotional assessments \cite{nidalEEGERPAnalysis2014}. More specifically, this framework can be coupled with basic neuroscience paradigms to study the neural correlates of emotions.

\subsection{Neuroscience of Neurourbanism}

As the name suggests neuroubanism implies the use of different functional neuroimaging and neurophysiological tools, such as the EEG, functional magnetic resonance imaging (fMRI), and functional near-infrared spectrometry (fNIRS). By combining neurological measurements with behavioral and physiological data, researchers can establish robust correlations between specific urban characteristics and their impact on sensory processing, cognition, and emotional states. Several studies have already employed this approach in the context of neurourbanism whose general goal is to understand how neural measures relate to the city environments and related urban characteristics influence our brain activity \cite{ancoraCitiesNeuroscienceResearch2022, berkmanBrainMapping2013}.

%%%%%%%%%%%%%%%%%%%%%%%%%%%%%%%%%%%%%%%%%%%%%%%%%%%%%%%%%%%%%%%%%%%%%%%%
\section{Electrophysiology of Emotions}
\label{section:overview}

\subsection{EEG features}

Among the available neuroimaging techniques, EEG has emerged as the preferred method for studying brain activity in urban contexts \cite{ancoraCitiesNeuroscienceResearch2022}. This preference stems from several key advantages: EEG offers high temporal resolution, comes with a comprehensive analytical toolset, can be used portably in real-world settings, and is relatively cost-effective \cite{ismailApplicationsEEGIndices2020}. 

In neurourbanism research, frequency-domain analysis of EEG data has been particularly revealing, with studies consistently showing increased alpha-band power during exposure to natural versus urban-built environments. For example, the well-established relationship between asymmetric frontal cortical activity and emotional valence has made EEG particularly valuable for assessing emotional responses to urban environments \cite{nidalEEGERPAnalysis2014}.

\subsection{ERP studies}

Complementing these frequency-domain findings, Event-Related Potential (ERP) analysis provides another valuable approach for examining time-locked emotional responses to environmental stimuli \cite{rahmanRecognitionHumanEmotions2021}. ERP studies typically seek to identify specific components that have been previously associated with particular cognitive processes (figure \ref{fig:erp_components}) \cite{donoghueAutomatedMetaanalysisEventrelated2022}. To ensure reliability and reproducibility, affective ERP research often combines well-established experimental paradigms, such as the oddball task, with standardized stimulus sets like the International Affective Picture System (IAPS) \cite{lang1999international}. These standardized stimuli are typically rated on valence and arousal scales, enabling researchers to identify consistent ERP modulations associated with different emotional responses \cite{olofssonAffectivePictureProcessing2008}. However, it's important to note that while this standardization enhances experimental robustness, the identified ERP components may not necessarily generalize to other tasks or types of environmental stimuli.

\begin{figure}[H]
	\centering
	\includegraphics[width=\linewidth]{Figures/erp_components.png}
	\caption{Different ERP components and their associations with cognitive-related states in the literature. The EPN and LPP components are highlighted as being the most associated with emotion, valence, and arousal. Figure adapted from \cite{donoghueAutomatedMetaanalysisEventrelated2022}. \label{fig:erp_components}}
\end{figure}

Two of the most well-studied ERPs associated with emotional processing are the early posterior negativity (EPN) and the late positive potential (LPP). These components have been reliably found during naturalistic scene perception, perception of emotional faces, hand gestures, and words \cite{farkasEmotionalPerceptionDivergence2023, farkasEmotionalFeaturebasedModulation2020}. This robustness to a variety of stimulus suggests independence from any particular evocative modality and motivates the study of these components in the context of urbanism.

The \acrfull{lpp} is an ERP component that is mostly studied with respect to stimulus valence. It is characterized by being a cerebral midline ERP (i.e., having a central-parietal scalp distribution) that becomes evident approximately 300 milliseconds following stimulus onset; and it is greater following the presentation of both pleasant and unpleasant compared to neutral pictures and words (part a) of figure \ref{fig:lpp_epn}). The LPP has also been found to be increased for intense stimuli. As a whole, it is a suitable ERP component to study the effects of natural and urban-built scenery on emotion-related responses such as valence and arousal \cite{bradleyMeasuringEmotionSelfassessment1994, hajcakEventRelatedPotentialsEmotion2010, kappenmanERPComponentsUps2011}.

The \acrfull{epn} appears as a negative voltage deflection over left and right lateral occipital sensors from 150 to 300 msec after stimulus onset (part b) of figure \ref{fig:lpp_epn}). The EPN is modulated by the arousal of the scene and is thought to reflect an early stage of enhanced perceptual processing associated with motivated attention \cite{junghoferFleetingImagesNew2001, farkasEmotionalPerceptionDivergence2023}.

\begin{figure}[H]
	\centering
	\includegraphics[width=\linewidth]{Figures/lpp_epn.png}
	\caption{\textbf{a)} LPP component and its modulation by stimulus valence. Figure taken from \cite{hajcakEventRelatedPotentialsEmotion2010}. \textbf{b)} The EPN and its modulation by stimulus arousal. Figure taken from \cite{junghoferFleetingImagesNew2001}. \label{fig:lpp_epn}}
\end{figure}

\subsection{ERP source localization}

EEG researchers recognize that the signals recorded at the scalp cannot be directly equated to their neural origins. Event-Related Potentials (ERPs), while extensively studied, represent high-level phenomena that require further investigation into their underlying neural mechanisms. Source reconstruction techniques serve as a critical bridge, linking scalp-recorded signals to their neural generators (see subsection \ref{section:ESI}). These methods have become particularly effective with high-density EEG (HD-EEG) systems, enabling the estimation of neural origins for specific ERP components, including early components (e.g., P1-N1 complex, N170, P2, N2) and later components (e.g., MMN, P300, N400) \cite{dattolaFindingsLORETAApplied2020}.

For instance, source localization has been employed to trace the neural generators of the P300 component during exposure to advertisements with varying levels of arousal \cite{daughertyMeasuringConsumerNeural2018}. In clinical research, LORETA source analysis of the \acrfull{mmn} and P300 components has provided valuable insights into early neurodegenerative changes, distinguishing patterns in patients with \acrfull{mci} and \acrfull{ad} \cite{tsolakiAgeinducedDifferencesBrain2017}.

Despite the increasing prevalence of ERP localization studies and the adoption of HD-EEG systems (with over 128 channels), many studies still rely on unoptimized template head models, highlighting an area for methodological improvement.

%%%%%%%%%%%%%%%%%%%%%%%%%%%%%%%%%%%%%%%%%%%%%%%%%%%%%%%%%%%%%%%%%%%%%%%%
\section{Objectives}
\label{section:objectives}

\subsection{General Objectives}

The primary objective of this study is to apply source estimation techniques to ERP data from two contrasting conditions (natural versus urban-built stimuli) to identify differences in both the time-domain and source-domain.

\subsection{Specific Objectives}

\begin{enumerate}
	\item Conduct high-density EEG analysis in a controlled lab-based paradigm involving visual presentations of natural and urban-built environments.
	
	\item Investigate whether the EPN and LPP components, which are described in the literature as markers of emotion-related responses, can be observed during the passive viewing of urban scenarios and whether these components differ between natural and urban-built conditions.
	
	\item Apply electrical source imaging (ESI) to evaluate differences in source activity across conditions and identify the anatomical regions contributing to these differences.
\end{enumerate}